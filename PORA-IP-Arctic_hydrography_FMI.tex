\documentclass{beamer}

\begin{document}
  \begin{frame}
    \frametitle{ORA products used for hydrographic analysis}
    \begin{itemize}
        \item Observational
        \begin{itemize}
        \item Climatology by Hiroshi Sumata (AWI) 1980--2015.
        \item WOA13 1995--2012. 
        \end{itemize}
        \item ORA products, 1993--2010.
        \begin{itemize}
        \item GloSea5$^*$, GECCO2 (4DVAR), GLORYS2V4$^*$ and CGLORS$^*$.
        \item UoR$^*$, ORAP5$^*$ and TOPAZ$^*$.
        \item MOVEG2, EN4 (objective analysis) and ECDA.
        \end{itemize}
    \end{itemize}
    ($^*$) Indicates products with sea-ice observation assimilation.
  \end{frame}
  \begin{frame}
    \frametitle{Notes on diagnostics}
    \begin{columns}
    \begin{column}{0.48\textwidth}
    \includegraphics[width=\columnwidth]{{PORA-IP-Arctic-points}.png}
    \begin{itemize}
    \item Three locations (above) selected for the analysis (Amerasian, central Arctic and Eurasian).
    \item T and S calculated from layer-averaged heat and salinity content data.
    \end{itemize}
    \end{column}
    \begin{column}{0.48\textwidth}
    \begin{itemize}
    \item Five layers: 0--100, 100--300, 300--700, 700--1500 and 1500--3000 m.
    \item T and S profiles are compared plus TS-diagrams.
    \item A transect across Fram Strait is analysed.
    \end{itemize}
    \end{column}
    \end{columns}
  \end{frame}
  \begin{frame}
    \frametitle{Temperature profiles from central Arctic.}
    \includegraphics[width=0.8\textwidth]{{T_UoR_GloSea5_MOVEG2_GECCO2_EN4_ECDA_ORAP5_TOPAZ_GLORYS_CGLORS_1993-2010_88.0N_010.0E}.pdf}
    \begin{itemize}
    \item MOVEG2 is very cold, except in 0--100 m where it is warm as is TOPAZ.
    \item GECCO2 has the warmest Atlantic layer (300--700 m), where TOPAZ and GloSea5 are closest to Sumata and WOA13.
    \end{itemize}
  \end{frame}
  \begin{frame}
    \frametitle{Salinity profiles from central Arctic.}
    \includegraphics[width=0.8\textwidth]{{S_UoR_GloSea5_MOVEG2_GECCO2_EN4_ECDA_ORAP5_TOPAZ_GLORYS_CGLORS_1993-2010_88.0N_010.0E}.pdf}
    \begin{itemize}
    \item GLORYS2V4 the saltiest model in top 300 m.
    \item CGLORS and TOPAZ have rather low salinity in top 100 m.
    \item At deeper than 300 m, models agree rather well except MOVE2 which is saltier than others.
    \end{itemize}
  \end{frame}
  \begin{frame}
    \frametitle{TS diagrams from central Arctic.}
    \includegraphics[width=0.8\textwidth]{{TS_UoR_GloSea5_MOVEG2_GECCO2_EN4_ECDA_ORAP5_TOPAZ_GLORYS_CGLORS_1993-2010_88.0N_010.0E}.pdf}
    \begin{itemize}
      \item 0--100 m (1): CGLORS freshest, GLORYS saltiest. MOVEG2 and TOPAZ warmest.
      \item 100--300 m (2): CGLORS much colder than others.
      \item 300--700 m (3): GECCO2 warmest, GLORYS coldest.
      \item 1500--3000 m (5): GLORYS warmest, MOVEG2 and GECCO2 coldest.
    \end{itemize}
  \end{frame}
  \begin{frame}
    \frametitle{Temperature profiles from Nansen Basin.}
    \includegraphics[width=0.8\textwidth]{{T_UoR_GloSea5_MOVEG2_GECCO2_EN4_ECDA_ORAP5_TOPAZ_GLORYS_CGLORS_1993-2010_84.0N_100.0E}.pdf}
    \begin{itemize}
    \item In 100--700 m Sumata is clearly warmer than WOA13 and ORAs. More observations in Sumata?
    \item ORAP5, GloSea5 and UoR have warm Atlantic layers.
    \item MOVEG2 is very cold, except in 0--100 m where it is warm, as is GLORYS.
    \end{itemize}
  \end{frame}
  \begin{frame}
    \frametitle{Salinity profiles from Nansen Basin.}
    \includegraphics[width=\textwidth]{{S_UoR_GloSea5_MOVEG2_GECCO2_EN4_ECDA_ORAP5_TOPAZ_GLORYS_CGLORS_1993-2010_84.0N_100.0E}.pdf}
    \begin{itemize}
    \item CGLORS is very fresh in 0--100 m .
    \end{itemize}
  \end{frame}
  \begin{frame}
    \frametitle{TS diagrams from Nansen Basin.}
    \includegraphics[width=\textwidth]{{TS_UoR_GloSea5_MOVEG2_GECCO2_EN4_ECDA_ORAP5_TOPAZ_GLORYS_CGLORS_1993-2010_84.0N_100.0E}.pdf}
    \begin{itemize}
      \item 0--100 m (1): EN4 is coldest, MOVEG2 and CLOGRS warmest.
    \end{itemize}
  \end{frame}
  \begin{frame}
    \frametitle{Temperature profiles from Amerasian Basin.}
    \includegraphics[width=0.8\textwidth]{{T_UoR_GloSea5_MOVEG2_GECCO2_EN4_ECDA_ORAP5_TOPAZ_GLORYS_CGLORS_1993-2010_80.0N_215.0E}.pdf}
    \begin{itemize}
    \item WOA and Sumata agree well except below 700 m where Sumata is warmer.
    \item ORAs are colder than WOA and Sumata in 300--700 m.
    \item In 0--100 m GECCO2 is coldest, while  below 300 m MOVEG2 is very cold.
    \item ECDA is missing the warm water layer in 300--700 m.
    \end{itemize}
  \end{frame}
  \begin{frame}
    \frametitle{Salinity profiles from Amerasian Basin.}
    \includegraphics[width=\textwidth]{{S_UoR_GloSea5_MOVEG2_GECCO2_EN4_ECDA_ORAP5_TOPAZ_GLORYS_CGLORS_1993-2010_80.0N_215.0E}.pdf}
    \begin{itemize}
    \item MOVEG2 is very salty in 0--300 m.
    \end{itemize}
  \end{frame}
  \begin{frame}
    \frametitle{TS diagrams from Amerasian Basin.}
    \includegraphics[width=\textwidth]{{TS_UoR_GloSea5_MOVEG2_GECCO2_EN4_ECDA_ORAP5_TOPAZ_GLORYS_CGLORS_1993-2010_80.0N_215.0E}.pdf}
    \begin{itemize}
    \item Sumata is closer to WOA13 than in the Nansen basin.
    \end{itemize}
  \end{frame}
  \begin{frame}
    \frametitle{Fram Strait temperature transect.}
    \begin{columns}
    \begin{column}{0.6\textwidth}
    \includegraphics[width=\textwidth]{{fstransect_T}.png}
    \end{column}
    \begin{column}{0.4\textwidth}
    \begin{itemize}
    \item Generally realistic: cold water in the west and warm in the west.
    \item Differences in terms of the strength and extent of warm/cold core water.
    \item Need to look at inter-product differences more in detail using anomalies, for instance. 
    \end{itemize}
    \end{column}
    \end{columns}
  \end{frame}
  \begin{frame}
    \frametitle{Fram Strait salinity transect.}
    \begin{columns}
    \begin{column}{0.6\textwidth}
    \includegraphics[width=\textwidth]{{fstransect_S}.png}
    \end{column}
    \begin{column}{0.4\textwidth}
    \begin{itemize}
    \item Some masking issues which need to be polished out.
    \item The amount of fresh water varies between products.
    \end{itemize}
    \end{column}
    \end{columns}
  \end{frame}
  \begin{frame}
    \frametitle{General remarks}
    \begin{itemize}
    \item In the Nansen basin in 100--700 m Sumata is warmer than WOA13.
    \item Differences larger in top 700 m, but better agreement in deep.
    \item Width and temperature of the warm Atlantic water layer varies between ORAs.
    \item Compared to observed, ORAs have too cold Atlantic layer.
    \end{itemize}
  \end{frame}

\end{document}
