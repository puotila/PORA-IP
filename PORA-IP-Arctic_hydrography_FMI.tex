\documentclass{beamer}

\begin{document}
  \begin{frame}
    \frametitle{ORA products used for hydrographic analysis}
    \begin{itemize}
        \item Observational
        \begin{itemize}
        \item Climatology by Hiroshi Sumata (AWI) 1980--2015.
        \item WOA13 1995--2004. 
        \end{itemize}
        \item ORA products, 1993--2009.
        \begin{itemize}
        \item CGLORS$^*$, GECCO2 (4DVAR), GLORYS2V4$^*$ and GloSea5$^*$.
        \item ORAP5$^*$, TP4$^*$ and UoR$^*$.
        \item ECDA, MOVEG2, EN4 (objective analysis).
        \end{itemize}
    \end{itemize}
    ($^*$) Indicates products with sea-ice observation assimilation.
  \end{frame}
  \begin{frame}
    \frametitle{Notes on diagnostics}
    \begin{columns}
    \begin{column}{0.48\textwidth}
    \includegraphics[width=\columnwidth]{{PORA-IP-Arctic-points}.png}
    \begin{itemize}
    \item Three locations (above) selected for the analysis (Amerasian, Central Arctic and Eurasian basin).
    \item T and S calculated from layer-averaged heat and salinity content data.
    \end{itemize}
    \end{column}
    \begin{column}{0.48\textwidth}
    \begin{itemize}
    \item Seven layers: 0--10, 10--100, 100--300, 300--700, 700--1500, 1500--3000 and 3000--4000 m.
    \item T and S profiles are compared plus TS-diagrams.
    \item Two representative locations, one in the Nansen Basin and another in the AmerAsian Basin are selected (todo).
    \item A transect across Fram Strait is analysed (todo).
    \item In TS-diagrams 0--10 m layer is not presented.
    \end{itemize}
    \end{column}
    \end{columns}
  \end{frame}
  \begin{frame}
    \frametitle{Temperature profiles from central Arctic.}
    \includegraphics[width=\textwidth]{{T_Sumata_WOA13_CGLORS_ECDA_GloSea5_GO5_MOVEG2_UoR_EN4.2.0.g10_GECCO2_GSOP_GLORYS2V4_ORAP5_TP4_1980-2015_88.0N_010.0E}.png}
    \begin{itemize}
    \item Sumata fresher than WOA13 in 0-100m.
    \item GLORYS2V4 the saltiest model, then MOVEG2.
    \item TP4 the freshest model product.
    \end{itemize}
  \end{frame}
  \begin{frame}
    \frametitle{Salinity profiles from central Arctic.}
    \includegraphics[width=0.8\textwidth]{{S_Sumata_WOA13_CGLORS_ECDA_GloSea5_GO5_MOVEG2_UoR_EN4.2.0.g10_GECCO2_GSOP_GLORYS2V4_ORAP5_TP4_1980-2015_88.0N_010.0E}.png}
    \begin{itemize}
    \item MOVEG2 is very cold.
    \item Sumata surface is much warmer than WOA13 and ORA surface temperatures closer to WOA13.
    \item GECCO2 has the warmest Atlantic layer, TP4 closest to Sumata and WOA13 in the Atlantic layer (300--700 m).
    \item TP4 is closest to Sumata and WOA13 in the Atlantic layer.
    \end{itemize}
  \end{frame}
  \begin{frame}
    \frametitle{TS diagrams from central Arctic.}
    \includegraphics[width=\textwidth]{{ST_Sumata_WOA13_CGLORS_ECDA_GloSea5_GO5_MOVEG2_UoR_EN4.2.0.g10_GECCO2_GSOP_GLORYS2V4_ORAP5_TP4_1980-2015_88.0N_010.0E}.png}
    \begin{itemize}
    \item Sumata TS at surface is very warm and fresh compared to others.
    \item MOVEG2 seems also quite different than others.
    \end{itemize}
  \end{frame}
  \begin{frame}
    \frametitle{Temperature profiles from Nansen Basin.}
    \includegraphics[width=0.8\textwidth]{{T_Sumata_WOA13_CGLORS_ECDA_GloSea5_GO5_MOVEG2_UoR_EN4.2.0.g10_GECCO2_GSOP_GLORYS2V4_ORAP5_TP4_1980-2015_83.0N_100.0E}.png}
    \begin{itemize}
    \item MOVEG2 and TP4 are cold in top 300 m.
    \item Generally ORAs are colder than Sumata and WOA13 in the Atlantic layer.
    \item At surface Sumata is warmer than WOA13.
    \item ORAP5 has the warmest Atlantic layer, UoR has a wide one and GLORYS2V4 a rather cold one.
    \end{itemize}
  \end{frame}
  \begin{frame}
    \frametitle{Salinity profiles from Nansen Basin.}
    \includegraphics[width=\textwidth]{{S_Sumata_WOA13_CGLORS_ECDA_GloSea5_GO5_MOVEG2_UoR_EN4.2.0.g10_GECCO2_GSOP_GLORYS2V4_ORAP5_TP4_1980-2015_83.0N_100.0E}.png}
    \begin{itemize}
    \item Sumata is much fresher than WOA13 in top 100 m.
    \item Model salinities are between WOA13 and Sumata.
    \end{itemize}
  \end{frame}
  \begin{frame}
    \frametitle{TS diagrams from Nansen Basin.}
    \includegraphics[width=\textwidth]{{ST_Sumata_WOA13_CGLORS_ECDA_GloSea5_GO5_MOVEG2_UoR_EN4.2.0.g10_GECCO2_GSOP_GLORYS2V4_ORAP5_TP4_1980-2015_83.0N_100.0E}.png}
    \begin{itemize}
    \item Sumata is rather different from others.
    \end{itemize}
  \end{frame}
  \begin{frame}
    \frametitle{Temperature profiles from Amerasian Basin.}
    \includegraphics[width=0.8\textwidth]{{T_Sumata_WOA13_CGLORS_ECDA_GloSea5_GO5_MOVEG2_UoR_EN4.2.0.g10_GECCO2_GSOP_GLORYS2V4_ORAP5_TP4_1980-2015_80.0N_220.0E}.png}
    \begin{itemize}
    \item In top 700 m, ORAs are colder than WOA13 and Sumata. In top 300 m, Sumata is very warm compared to WOA13.
    \item MOVEG2 is the coldest; ECDA is missing the warm water entirely; UoR, GECCO2 and EN4 are warmest.
    \item Agreement is reasonable below 700m.
    \end{itemize}
  \end{frame}
  \begin{frame}
    \frametitle{Salinity profiles from Amerasian Basin.}
    \includegraphics[width=\textwidth]{{S_Sumata_WOA13_CGLORS_ECDA_GloSea5_GO5_MOVEG2_UoR_EN4.2.0.g10_GECCO2_GSOP_GLORYS2V4_ORAP5_TP4_1980-2015_80.0N_220.0E}.png}
    \begin{itemize}
    \item Sumata is the saltiest, GECCO2 the freshest (100--300 m).
    \item In 10--100 m, WOA13, EN4, GloSea5, ORAP5 and UoR are rather fresh.
    \end{itemize}
  \end{frame}
  \begin{frame}
    \frametitle{TS diagrams from Amerasian Basin.}
    \includegraphics[width=\textwidth]{{ST_Sumata_WOA13_CGLORS_ECDA_GloSea5_GO5_MOVEG2_UoR_EN4.2.0.g10_GECCO2_GSOP_GLORYS2V4_ORAP5_TP4_1980-2015_80.0N_220.0E}.png}
    \begin{itemize}
    \item Sumata somewhat closer to others than in Eurasian basin, although still colder and fresher in the surface layer.
    \end{itemize}
  \end{frame}
  \begin{frame}
    \frametitle{General remarks}
    \begin{itemize}
    \item Differences large in top 700 m, better agreement deeper.
    \item ORAs seem to have too cold Atlantic layer.
    \item Sumata climatology is colder and fresher from ORAs and WOA13, particularly in upper layers.
    \end{itemize}
  \end{frame}

\end{document}
